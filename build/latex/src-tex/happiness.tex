
The goal of Buddhist meditation is Nibb\=ana. We incline towards the peace of Nibb\=ana and away from the complexities of the sensual realm, the endless cycles of habit. Nibb\=ana is a goal that can be realized in this lifetime. We don't have to wait until we die to know if it's real.

The senses and the sensual world are the realm of birth and death. Take sight, for instance: it's dependent on so many factors -- whether it's day or night, whether or not the eyes are healthy, and so on. Yet we become very attached to the colours, shapes and forms that we perceive with the eyes, and we identify with them. Then there are the ears and sound: when we hear pleasant sounds we seek to hold on to them, and when we hear unpleasant sounds we try to turn away. With smells we seek the pleasure of fragrances and pleasant odours, and try to get away from unpleasant ones. Also with flavours: we seek delicious tastes and try to avoid bad ones. And with touch: just how much of our life is spent trying to escape from physical discomfort and pain and seeking the delight of physical sensation? Finally there is thought, the discriminative consciousness; it can give us a lot of pleasure or a lot of misery.

\looseness=1
These are the senses, the sensual world. It is the compounded world of birth and death. Its very nature is \textit{dukkha}, it is imperfect and unsatisfying. You'll never find perfect happiness, contentment or peace in the sensual world; it will always bring despair and death. The sensual world is unsatisfactory, and so we only suffer from it when we expect it to satisfy us. We suffer from it when we expect more from it than it can possibly give, things like permanent security and happiness, permanent love and safety, hoping that our life will only be one of pleasure and have no pain in it. `If we could only get rid of sickness and disease and conquer old age.'

I remember 20 years ago in the States people had this great hope that modem science would be able to get rid of all illnesses. They'd say, `All mental illnesses are due to chemical imbalances. If we can just find the right chemical combinations and inject them into the body, schizophrenia will disappear.' There would be no more headaches or backaches. We would gradually replace all our internal organs with nice plastic ones. I even read an article in an Australian medical journal about how they hoped to conquer old age! As the world's population keeps increasing, we'd keep having more children and nobody would ever get old and die. Just think what a mess that would be!

The sensual world is unsatisfactory and that's the way it's supposed to be. When we attach to it, it takes us to despair -- because attachment means that we want it to be satisfactory, we want it to satisfy us, to make us content, happy and secure. But just notice the nature of happiness -- how long can you stay happy? What is happiness? You may think it's how you feel when you get what you want. Someone says something you like to hear, and you feel happy. Someone does something you approve of, and you feel happy. The sun shines and you feel happy. Someone makes nice food and serves it to you, and you're happy. But how long can you stay happy? Do we always have to depend on the sun shining? In England the weather is very changeable: happiness about the sun shining in England is obviously very impermanent and unsatisfactory!

Unhappiness is not getting what we want: wanting it to be sunny when it's cold, wet and rainy; people doing things that we don't approve of; having food that isn't delicious, and so on. Life gets boring and tedious when we're unhappy with it. So happiness and unhappiness are very dependent on getting what we want, or getting what we don't want.

But happiness is the goal of most people's lives; the American constitution, I think, speaks of `the right to the pursuit of happiness'. Getting what we want, what we think we deserve, becomes our goal in life. But happiness always leads to unhappiness, because it's impermanent. How long can you really be happy? Trying to arrange, control and manipulate conditions so as always to get what we want, always hear what we want to hear, always see what we want to see, so that we never have to experience unhappiness or despair, is a hopeless task. It's impossible, isn't it? Happiness is unsatisfactory, it's \textit{dukkha}. It's not something to depend on or make the goal of life. Happiness will always be disappointing, because it lasts so briefly and then is succeeded by unhappiness. It is always dependent on so many other things. We feel happy when we're healthy, but our human bodies are subject to rapid changes and we can lose that health very quickly. Then we feel terribly unhappy at being sick, at losing the pleasure of feeling energetic and vigorous.

Thus the goal for the Buddhist is not happiness, because we realize that happiness is unsatisfactory. The goal lies away from the sensual world. It is not rejection of the sensual world, but understanding it so well that we no longer seek it as an end in itself. We no longer expect the sensory world to satisfy us. We no longer demand that sensory consciousness be anything other than an existing condition that we can use skilfully according to time and place. We no longer attach to it, or demand that the sense impingement be always pleasant, or feel despair and sorrow when it's unpleasant. Nibb\=ana isn't a state of blankness, a trance where you're totally wiped out. It's not nothingness or annihilation: it's like a space. It's going into the space of your mind where you no longer attach, where you're no longer deluded by the appearance of things. You are no longer demanding anything from the sensory world. You are just recognizing it as it arises and passes away.

Being born in the human condition means that we must inevitably experience old age, sickness and death. One time a young woman came to our monastery in England with her baby. The baby had been badly ill for about a week with a horrible racking cough. The mother looked totally depressed and miserable. As she sat there in the reception room holding the baby, it fumed red in the face and started screaming and coughing horribly. The woman said, `Oh, Venerable Sumedho, why does he have to suffer like this? He's never hurt anybody, he's never done anything wrong. Why? In some previous life, what did he do to have to suffer like this?' He was suffering because he was born! If he hadn't been born, he wouldn't have had to suffer. When we're born we have to expect these things. Having a human body means that we have to experience sickness, pain, old age and death. This is an important reflection. We can speculate that maybe in a previous life he liked to choke cats and dogs or something like that, and he has to pay for it in this life, but that's mere speculation and it doesn't really help. What we can know is that it's the kammic result of being born. Each one of us must inevitably experience sickness and pain, hunger, thirst, the ageing process of our bodies and death -- it's the law of kamma. What begins must end; what is born must die; what comes together must separate.

We're not being pessimistic about the way things are, but we're observing, so we don't expect life to be other than it is. Then we can cope with life and endure it when it's difficult, and delight in it when it's delightful. If we understand life, we can enjoy it without being its helpless victims. How much misery there is in human existence because we expect life to be other than what it is! We have these romantic ideas that we'll meet the right person, fall in love and live happily ever after, that we'll never fight and have a wonderful relationship. But what about death? So you think, `Well, maybe we'll die at the same time.' That's hope, isn't it? There's hope, and then despair when your loved one dies before you do, or runs away with the dustman or the travelling salesman.

You can learn a lot from small children, because they don't disguise their feelings, they just express what they feel in the moment; when they're miserable they start crying, and when they're happy they laugh. Some time ago I went to a layman's home. When we arrived, his young daughter was very happy to see him. Then he said to her, `I have to take Venerable Sumedho to Sussex University to give a talk.' As we walked out of the door, the little girl fumed red in the face and began screaming in anguish, so her father said, `It's all right, I'll be back in an hour.' But she wasn't developed to that level where she could understand `I'll be back in an hour.' The immediacy of separation from the loved meant immediate anguish.

Notice how often in our life there is that sorrow at having to separate from something we like or someone we love, from having to leave a place we really like to be in. When you are really mindful you can see the not wanting to separate, the sorrow. As adults, we can let go of it immediately if we know we can come back again, but it's still there. From last November to March, I travelled around the world, always arriving at airports with somebody meeting me with a `Hello!' -- and then a few days later it was `Goodbye!' And there was always this sense of `Come back', and I'd say `Yes, I'll come back'... and so I've committed myself to do the same thing next year. We can't say, `Goodbye forever' to someone we like, can we? We say, `I'll see you again,' `I'll phone you up,' `I'll write you a letter', or `Until next time we meet'. We have all these phrases to cover over the sense of sorrow and separation.

In meditation we're noting, just observing what sorrow really is. We're not saying that we shouldn't feel sorrow when we separate from someone we love; it's natural to feel that way, isn't it? But now, as meditators, we're beginning to witness sorrow so that we understand it, rather than trying to suppress it, pretend it's something more than it is, or just neglect it.

\looseness=1
In England people tend to suppress sorrow when somebody dies. They try not to cry or be emotional, they don't want to make a scene, they `keep a stiff upper lip'. Then when they start meditating they can find themselves suddenly crying over the death of someone who died fifteen years before. They didn't cry at the time, so they end up doing it fifteen years later. When someone dies we don't want to admit the sorrow or make a scene, because we think that if we cry we're weak, or it's embarrassing to others. So we tend to suppress and hold things back, not recognizing the nature of things as they really are, not recognizing our human predicament and learning from it. In meditation we're allowing the mind to open up and let the things that have been suppressed and repressed become conscious, because when things become conscious they have a way of ceasing rather than just being repressed again. We allow things to take their course to cessation, we allow things to go away rather than just push them away. Usually we just push certain things away from us, refusing to accept or recognize them. Whenever we feel upset or annoyed with anyone, when we're bored or when unpleasant feelings arise, we look at the beautiful flowers or the sky, read a book, watch TV, do something. We're never fully consciously bored, fully angry. We don't recognize our despair or disappointment, because we can always run off into something else. We can always go to the refrigerator, eat cakes and sweets, listen to the stereo. It's so easy to absorb into music, away from boredom and despair into something that's exciting or interesting or calming or beautiful. Look at how dependent we are on watching TV and reading. There are so many books now that they'll all have to be burnt -- useless books everywhere, everybody's writing things without having anything worth saying. Today's not-so-pleasant film stars write their biographies and make a lot of money. Then there are the gossip columns: people get away from the boredom of their own existence, their discontent with it, the tediousness, by reading gossip about movie stars and public figures.

We've never really accepted boredom as a conscious state. As soon as it comes into the mind we start looking for something interesting, something pleasant. But in meditation we're allowing boredom to be. We're allowing ourselves to be fully consciously bored, fully depressed, fed up, jealous, angry, disgusted. All the nasty, unpleasant experiences of life that we have repressed out of consciousness and never really looked at, never really accepted, we begin to accept into consciousness -- not as personality problems any more, but just out of compassion. Out of kindness and wisdom we allow things to take their natural course to cessation, rather than just keeping them going round in the same old cycles of habit. If we have no way of letting things take their natural course, then we're always controlling, always caught in some dreary habit of mind. When we're jaded and depressed we're unable to appreciate the beauty of things, because we never really see them as they truly are.

I remember one experience I had in my first year of meditation in Thailand. I spent most of that year by myself in a little hut, and the first few months were really terrible. All kinds of things kept coming up in my mind -- obsessions and fears and terror and hatred. I'd never felt so much hatred. I'd never thought of myself as one who hated people, but during those first few months of meditation it seemed like I hated everybody. I couldn't think of anything nice about anyone, there was so much aversion coming up into consciousness. Then one afternoon I started having this strange vision -- I thought I was going crazy, actually -- I saw people walking off my brain. I saw my mother just walk out of my brain and into emptiness, disappear into space. Then my father and my sister followed. I actually saw these visions walking out of my head. I thought, `I'm crazy! I've gone off!' -- but it wasn't an unpleasant experience.

The next morning, when I woke from sleep and looked around, I felt that everything I saw was beautiful. Everything, even the most unbeautiful detail, was beautiful. I was in a state of awe. The hut itself was a crude structure, not beautiful by anyone's standards, but it looked to me like a palace. The scrubby-looking trees outside looked like a most beautiful forest. Sunbeams were streaming through the window onto a plastic dish, and the plastic dish looked beautiful! That sense of beauty stayed with me for about a week and then, reflecting on it, I suddenly realized that's the way things really are when the mind is clear. Up to that time I'd been looking through a dirty window, and over the years I'd become so used to the scum and dirt on the window that I didn't realize it was dirty, I'd thought that's the way it was.

\looseness=1
When we get used to looking through a dirty window, everything seems grey, grimy and ugly. Meditation is a way of cleaning the window, purifying the mind, allowing things to come up into consciousness and letting them go. Then with the wisdom faculty, the Buddha-wisdom, we observe how things really are. This is not just attaching to beauty, to purity of mind, but actually understanding. It is wisely reflecting on the way nature operates, so that we are no longer deluded by it into creating habits for our life through \mbox{ignorance}.

Birth means old age, sickness and death, but that's to do with your body, it's not you. Your human body is not really yours. No matter what your particular appearance might be, whether you are healthy or sickly, whether you are beautiful or not beautiful, whether you are black or white or whatever, it's all non-self. This is what we mean by \textit{anatt\=a}, that human bodies belong to nature, that they follow the laws of nature: they are born, they grow up, they get old and they die. Now, we may understand that rationally, but emotionally there is a very strong attachment to the body. In meditation we begin to see this attachment. We don't take the position that we shouldn't be attached, saying, `The problem with me is that I'm attached to my body. I shouldn't be. It's bad, isn't it? If I was a wise person I wouldn't be attached to it.' That's starting from an ideal again. It's like trying to start climbing a tree from the top, saying, `I should be at the top of the tree. I shouldn't be down here.' But as much as we'd like to think that we're at the top, we have to accept humbly that we aren't. To begin with, we have to be at the trunk of the tree, where the roots are, looking at the most coarse and ordinary things, before we can start identifying with anything at the top of the tree.

This is the way of wise reflection. It's not just purifying the mind and then attaching to purity. It's not just trying to refine consciousness so that we can induce high states of concentration whenever we feel like it, because even the most refined states of sensory consciousness are unsatisfactory, they're dependent on so many other things. Nibb\=ana is not dependent on any other condition. Conditions of any quality, be they ugly, nasty, beautiful, refined or whatever, arise and pass away -- but they don't interfere with Nibb\=ana, with the peace of the mind.

We are not rejecting the sensory world out of aversion, because if we try to annihilate the senses, then that too becomes a habit that we blindly acquire, trying to get rid of that which we don't like. That's why we have to be very patient.

\looseness=1
This lifetime as a human being is a lifetime of meditation. See the rest of your life as the span of meditation rather than this ten-day retreat. You may think, `I meditated for ten days. I thought I was enlightened but somehow when I got home I didn't feel enlightened any more. I'd like to go back and do a longer retreat where I can feel more enlightened than I did last time. It would be nice to have a higher state of consciousness.' In fact, the more refined your experience, the more coarse your daily life must seem. You get high, and then when you get back to the mundane daily routines of life in the city, it's even worse than before, isn't it? After going so high, the ordinariness of life seems much more ordinary, gross and unpleasant. The way to insight wisdom is not by following preferences for refinement over coarseness, but recognizing that both refined and coarse consciousness are impermanent conditions, that they're unsatisfactory, their nature will never \mbox{satisfy} us, and they're \textit{anatt\=a}, they're not what we are, they're not~ours.

Thus the Buddha's teaching is a very simple one. What could be more simple than `what is born must die'? It's not some great new philosophical discovery; even illiterate tribal people know that. You don't have to study in university to know it.

When we're young we think, `I've got so many years left of youth and happiness.' If we're beautiful we think, `I'm going to be young and beautiful forever,' because it seems that way. If we're twenty years old, having a good time, life is wonderful and somebody says, `You are going to die some day', we may think, `What a depressing person. Let's not invite him to our house again.' We don't want to think about death, we want to think about how wonderful life is, how much pleasure we can get out of it.

But as meditators we reflect on getting old and dying. This is not being morbid or sick or depressing, but it's considering the whole cycle of existence; and when we know that cycle, then we are more careful about how we live. People do horrible things because they don't reflect on their deaths. They don't wisely reflect and consider, they just follow their passions and feelings of the moment, trying to get pleasure, and then feeling angry and depressed when life doesn't give them what they want.

Reflect on your own life and death and the cycles of nature. Just observe what delights and what depresses. See how we can feel very positive or very negative. Notice how we want to attach to beauty, or to pleasant feelings, or to inspiration. It's really nice to feel inspired, isn't it? `Buddhism is the greatest religion of them all', or `When I discovered the Buddha I was so happy, it's a wonderful discovery!' When we get a little bit doubtful, a little bit depressed, we go and read an inspiring book and get high. But remember, getting high is an impermanent condition; it's like becoming happy, you have to keep doing it, sustaining it, and after you keep doing something over and over again you no longer feel happy with it. How many sweets can you eat? At first they make you happy -- and then they make you sick.

So depending on religious inspiration is not enough. If you attach to inspiration, when you get fed up with Buddhism you'll go off and find some new thing to inspire you. It's like attaching to romance; when it disappears from the relationship you start looking for someone else to feel romantic towards. Years ago in America I met a woman who'd been married six times, and she was only about thirty-three. I said, `You'd think you would have learned after the third or fourth time. Why do you keep getting married?' She said, `It's the romance. I don't like the other side but I love the romance.' At least she was honest, but not terribly wise. Romance is a condition that leads to disillusionment.

Romance, inspiration, excitement, adventure: all these things rise to a peak and then condition their opposites, just as an inhalation conditions an exhalation. Just think of inhaling all the time. It's like having one romance after another, isn't it? How long can you inhale? The inhalation conditions the exhalation, both are necessary. Birth conditions death, hope conditions despair and inspiration conditions disillusionment. So when we attach to hope we're going to feel despair. When we attach to excitement it's going to take us to boredom. When we attach to romance it will take us to disillusionment and divorce. When we attach to life it takes us to death. So recognize that it's the attachment that causes the suffering, attaching to conditions and expecting them to be more than what they are.

So much of life for so many people seems to be waiting and hoping for something to happen -- expecting and anticipating some success or pleasure -- or maybe worrying and fearing that some painful, unpleasant thing is just lying in wait. You may hope that you will meet somebody you'll really love, or have some great experience, but attaching to hope takes you to despair.

\looseness=1
By wise reflection we begin to understand the things that create misery in our lives. We see that actually we are the creators of that misery. Through our ignorance, through not having wisely understood the sensory world and its limitations, we have identified with all that is unsatisfactory and impermanent, the things that can only take us to despair and death. No wonder life is so depressing! It's dreary because of the attachment, because we identify and seek ourselves in all that is by nature \textit{dukkha}: unsatisfactory and imperfect. Now, when we stop doing that, when we let go, that is enlightenment. We are enlightened beings, no longer attached, no longer identified with anything, no longer deluded by the sensory world. We understand the sensory world, we know how to co-exist with it. We know how to use the sensory world for compassionate action, for joyous giving. We don't demand that it be here to satisfy us any more, to make us feel secure and safe or to give us anything, because as soon as we demand that it should satisfy us, it takes us to despair.

When we no longer identify with the sensory world as `me' or `mine', and see it as \textit{anatt\=a}, we can enjoy the senses without seeking sense-impingement or depending on it. We no longer expect conditions to be anything other than what they are, so that when they change we can patiently and peacefully endure the unpleasant side of existence. We can humbly endure sickness, pain, cold, hunger, failures and criticisms. If we're not attached to the world we can adapt to change, whatever that change may be, whether it's for the better or for the worse. If we're still attached we can't adapt very well; we're always struggling, resisting, trying to control and manipulate everything, and then feeling frustrated, frightened or depressed at what a delusive, frightening place the world is. If you've never really contemplated the world, never taken the time to understand and know it, it becomes a frightening place for you. It becomes like a jungle: you don't know what's around the next tree, bush or cliff -- a wild animal, a ferocious man-eating tiger, a terrible dragon or a poisonous snake.

Nibb\=ana means getting away from the jungle. When we're inclining towards Nibb\=ana we're moving towards the peace of the mind. Although the conditions of the mind may not be peaceful at all, the mind itself is a peaceful place. Here we are making a distinction between the mind and the conditions of mind. The conditions of mind can be happy, miserable, elated, depressed, loving or hating, worrying or fear-ridden, doubting or bored. They come and go in the mind, but the mind itself, like the space in this room, stays just at it is. The space in this room has no quality to elate or depress, does it? It is just at it is. To concentrate on the space in the room we have to withdraw our attention from the things in it. If we concentrate on the things in the room we become happy or unhappy. We say, `Look at that beautiful Buddha image', or if we see something we find ugly we say, `Oh, what a terrible, disgusting thing.' We can spend our time looking at the people in the room, thinking whether we like this person or dislike that person. We can form opinions about people being this way or that way, remember what they did in the past, speculate about what they will do in the future, seeing others as possible sources of pain or gratification to ourselves. However, if we withdraw our attention it doesn't mean that we have to push everyone else out of the room. If we don't concentrate on or absorb into any of the conditions, then we have a perspective, because the space in the room has no quality to depress or elate. The space can contain us all, all conditions can come and go within it.

Moving inwards, we can apply this to the mind. The mind is like space, there's room in it for everything or nothing. It doesn't really matter whether it is filled or has nothing in it, because we always have a perspective once we know the space of the mind, its emptiness. Armies can come into the mind and leave, butterflies, rainclouds or nothing. All things can come and go through, without our being caught in blind reaction, struggling resistance, control or manipulation.

So when we abide in the emptiness of our minds we're moving away -- we're not getting rid of things, but no longer absorbing into conditions that exist in the present or creating any new ones. This is our practice of letting go. We let go of our identification with conditions by seeing that they are all impermanent and not-self. This is what we mean by \textit{vipassan\=a} meditation. It's really looking at, witnessing, listening, observing that whatever comes must go. Whether it's coarse or refined, good or bad, whatever comes and goes is not what we are. We're not good, we're not bad, we're not male or female, beautiful or ugly. These are changing conditions in nature, which are not-self. This is the Buddhist way to enlightenment: going towards Nibb\=ana, inclining towards the spaciousness or emptiness of mind rather than being born and caught up in the conditions.

Now you may ask, `Well if I'm not the conditions of mind, if I'm not a man or a woman, this or that, then what am I'? Do you want me to tell you who you are? Would you believe me if I did? What would you think if I ran out and started asking you who I am? It's like trying to see your own eyes: you can't know yourself, because you are yourself. You can only know what is not yourself -- and so that solves the problem, doesn't it? If you know what is not yourself, then there is no question about what you are. If I said, `Who am I? I'm trying to find myself,' and I started looking under the shrine, under the carpet, under the curtain, you'd think, `Venerable Sumedho has really flipped out, he's gone crazy, he's looking for himself.' `I'm looking for me, where am I?' is the most stupid question in the world. The problem is not who we are, but our belief and identification with what we are not. That's where the suffering is, that's where we feel misery and depression and despair. It's our identity with everything that is not ourselves that is \textit{dukkha}. When you identify with that which is unsatisfactory, you're going to be dissatisfied and discontented -- it's obvious, isn't it?

So the path of the Buddhist is a letting go rather than trying to find anything. The problem is the blind attachment, the blind identification with the appearance of the sensory world. You needn't get rid of the sensory world, but learn from it, watch it, no longer allow yourselves to be deluded by it. Keep penetrating it with Buddha-wisdom, keep using this Buddha-wisdom so that you become more at ease with being wise, rather than making yourself become wise. Just by listening, observing, being awake, being aware, the wisdom will become clear. You'll be using wisdom with regard to your body, with regard to your thoughts, feelings, memories, emotion, all of these things. You'll see and witness, allowing them to pass by and let them go.

So at this time you have nothing else to do except be wise from one moment to the next.
