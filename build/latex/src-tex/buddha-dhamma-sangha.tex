
When people ask, `What do you have to do to become a\linebreak\ Buddhist?', we say that we take refuge in Buddha, Dhamma, Sa\.ngha. And to take refuge we recite a formula in the P\=a\d{l}i language:

\begin{quote}
\textit{Buddha\d{m} sarana\d{m} gacch\=ami} \\
I go to the Buddha for refuge

\textit{Dhamma\d{m} sarana\d{m} gacch\=ami} \\
I go to the Dhamma for refuge

\textit{Sa\.ngha\d{m} sarana\d{m} gacch\=ami} \\
I go to the Sa\.ngha for refuge.
\end{quote}

As we practise more and more and begin to realize the profundity of the Buddhist teachings, it becomes a real joy to take these refuges, and even just their recitation inspires the mind. After twenty-two years as a monk, I still like to chant `\textit{Buddha\d{m} sarana\d{m} gacch\=ami}' -- in fact I like it more than I did twenty-one years ago, because then it didn't really mean anything to me, I just chanted it because I had to, because it was part of the tradition. Merely taking refuge verbally in the Buddha doesn't mean you take refuge in anything: a parrot could be trained to say `\textit{Buddha\d{m} sarana\d{m} gacch\=ami}', and it would probably be as meaningful to a parrot as it is to many Buddhists. These words are for reflection, looking at them and actually investigating what they mean: what `refuge' means, what `Buddha' means. When we say, `I take refuge in the Buddha,' what do we mean by that? How can we use that so it is not just a repetition of nonsense syllables, but something that really helps to remind us, gives us direction and increases our devotion, our dedication to the path of the Buddha?

The word `Buddha' is a lovely word -- it means `the one who knows' -- and the first refuge is in Buddha as the personification of wisdom. Un-personified wisdom remains too abstract for us: we can't conceive a bodiless, soul-less wisdom, and so as wisdom always seems to have a personal quality to it, using Buddha as its symbol is very useful.

We can use the word Buddha to refer to Gotama, the founder of what is now known as Buddhism, the historical sage who attained \textit{Parinibb\=ana} in India 2,500 years ago, the teacher of the Four Noble Truths and the Eightfold Path, teachings from which we still benefit today. But when we take refuge in the Buddha, it doesn't mean that we take refuge in some historical prophet, but in that which is wise in the universe, in our minds, that which is not separate from us but is more real than anything we can conceive with the mind or experience through the senses. Without any Buddha-wisdom in the universe, life for any length of time would be totally impossible; it is the Buddha-wisdom that protects. We call it Buddha-wisdom; other people can call it other things if they want, these are just words. We happen to use the words of our tradition. We're not going to argue about P\=a\d{l}i words, Sanskrit words, Hebrew, Greek, Latin, English or any other words, we're just using the term Buddha-wisdom as a conventional symbol to help remind us to be wise, to be alert, to be awake.

Many forest bhikkhus in the North-East of Thailand use the word `Buddho' as their meditation object. They use it as a kind of \textit{koan}. Firstly, they calm the mind by following the inhalations and exhalations using the syllables BUD-DHO, and then begin to contemplate, `What is Buddho, the ``one who knows''? What is the knowing?'

When I used to travel around the North-East of Thailand on \textit{tudong} (journeying), I liked to go and stay at the monastery of Ajahn Fun. Ajahn Fun was a much-loved and deeply respected monk, the teacher of the Royal Family, and he was so popular that he was constantly receiving guests. I would sit at his \textit{ku\d{t}\={\i}} (hut) and hear him give the most amazing kind of Dhamma talks, all on the subject of `Buddho' -- as far as I could see, it was all that he taught. He could make it into a really profound meditation, whether for an illiterate farmer or an elegant, western-educated Thai aristocrat. The main part of his teaching was  not just mechanically to repeat `Buddho', but to reflect and investigate, to awaken the mind really to look into the `Buddho', `the one who knows', really investigate its beginning, its end, above and below, so that one's whole attention was stuck onto it. When one did that, `Buddho' became something that echoed through the mind. One would investigate it, look at it, examine it before it was said and after it was said, and eventually one would start listening to it and hear beyond the sound, until one heard the silence.

A refuge is a place of safety, and so when superstitious people would come to my teacher Ajahn Chah, wanting\linebreak\ charmed medallions or little talismans to protect them from bullets, knives, ghosts and so on, he would say, `Why do you want things like that? The only real protection is taking refuge in the Buddha. Taking refuge in the Buddha is enough.' But their faith in Buddha usually wasn't quite as much as their faith in those silly little medallions. They wanted something made out of bronze and clay, stamped and blessed. This is what is called taking refuge in bronze and clay, taking refuge in superstition, taking refuge in that which is truly unsafe and cannot really help us.

Today in modern Britain we find that generally people are more sophisticated. They don't take refuge in magic charms, they take refuge in things like the Westminster Bank -- but that is still taking refuge in something that offers no safety. Taking refuge in the Buddha, in wisdom, means that we have a place of safety. When there is wisdom, when we act wisely and live wisely, we are truly safe. The conditions around us might change. We can't guarantee what will happen to the material standard of living, or that the Westminster Bank will survive the decade. The future remains unknown and mysterious, but in the present, by taking refuge in the Buddha we have that presence of mind now to reflect on and learn from life as we live it.

Wisdom doesn't mean having a lot of knowledge about the world; we don't have to go to university and collect information about the world to be wise. Wisdom means\linebreak\ knowing the nature of conditions as we're experiencing them. It is not just being caught up in reacting to and absorbing into the conditions of our bodies and minds out of habit, out of fear, worry, doubt, greed and so on, but it is using that `Buddho', that `one who knows,' to observe that these conditions are changing. It is the knowing of that change that we call Buddha and in which we take refuge. We make no claims to Buddha as being `me' or `mine'. We don't say, `I am Buddha,' but rather, `I take refuge in Buddha.' It is a way of humbly submitting to that wisdom, being aware, being awake.

Although in one sense taking refuge is something we are doing all the time, the P\=a\d{l}i formula we use is a reminder -- because we forget, because we habitually take refuge in worry, doubt, fear, anger, greed and so on. The Buddha-image is similar; when we bow to it we don't imagine that it is anything other than a bronze image, a symbol. It is a reflection and makes us a little more aware of Buddha, of our refuge in Buddha, Dhamma, Sa\.ngha. The Buddha image sits in great dignity and calm, not in a trance but fully alert, with a look of wakefulness and kindness, not caught in the changing conditions around it. Though the image is made of brass, and we have these flesh-and-blood bodies and it is much more difficult for us, still it is a reminder. Some people get very puritanical about Buddha-images, but here in the West I haven't found them to be a danger. The real idols that we believe in and worship, and that constantly delude us, are our thoughts, views and opinions, our loves and hates, our self-conceit and pride.

The second refuge is in the Dhamma, in ultimate truth or ultimate reality. Dhamma is impersonal; we don't in any way try to personify it, to make it any kind of personal deity. When we chant the verse on Dhamma in P\=a\d{l}i, we say it is \textit{`sandi\d{t}\d{t}hiko ak\=aliko ehipassiko opanayiko paccatta\d{m} veditabbo vi\~n\~n\=uhi'}. As Dhamma has no personal attributes, we can't even say it is good or bad, or anything that has any superlative or comparative quality; it is beyond the dualistic conceptions of mind.

So when we describe Dhamma or give an impression of it, we do it through words such as \textit{`sandi\d{t}\d{t}hiko'}, which means immanent, here-and-now. That brings us back into the present; we feel a sense of immediacy, of now. You may think that Dhamma is some kind of thing that is `out there', something you have to find elsewhere, but \textit{san\-di\d{t}\d{t}hiko\-dhamma} means that it is immanent, here-and-now.

\textit{Ak\=alikadhamma} means that Dhamma is not bound by any time condition. The word \textit{ak\=ala} means timeless. Our conceptual mind can't conceive of anything that is timeless, because our conceptions and perceptions are time-based conditions, but what we can say is that Dhamma is \textit{ak\=ala}, not bound by time.

\textit{Ehipassikadhamma} means to come and see, to turn towards or go to the Dhamma. It means to look, to be aware. It is not that we pray to the Dhamma to come, or wait for it to tap us on the shoulder; we have to put forth effort. It is like Christ's saying, `Knock on the door and it shall be opened.' \textit{Ehipassiko} means that we have to put forward that effort, to turn towards that truth.

\textit{Opanayiko} means leading inwards, towards peace within the mind. Dhamma doesn't take us into fascination, into excitement, romance or adventure, but leads to Nibb\=ana, to calm, to silence.

\textit{Paccatta\d{m} veditabbo vi\~n\~n\=uhi} means that we can only know Dhamma through direct experience. It is like the taste of honey -- if someone else tastes it, we still don't know its flavour. We may know the chemical formula or be able to recite all the great poetry ever written about honey, but only when we taste it for ourselves do we really know what it is like. It is the same with Dhamma: we have to taste it, we have to know it directly.

Taking refuge in Dhamma is taking another safe refuge. It is not taking refuge in philosophy or intellectual concepts, in theories, in ideas, in doctrines or beliefs of any sort. It is not taking refuge in a belief in Dhamma, or a belief in God, or in some kind of force in outer space or something beyond or something separate, something that we have to find some time later. The descriptions of the Dhamma keep us in the present, in the here and now, unbound by time. Taking refuge is an immediate, immanent reflection in the mind; it is not just repeating `\textit{Dhamma\d{m} sarana\d{m} gacch\=ami}' like a parrot, thinking, `Buddhists say this so I have to say it.' We turn towards the Dhamma, we are aware now, take refuge in Dhamma now, as an immediate action, an immediate reflection of being the Dhamma, being that very truth.

Because our conceiving mind tends always to delude us, it takes us into becoming. We think, `I'll practise meditation so that I'll become enlightened in the future. I will take the Three Refuges in order to become a Buddhist. I want to become wise. I want to get away from suffering and ignorance and become something else.' This is the conceiving mind, the desire mind, the mind that always deludes us. Rather than constantly thinking in terms of becoming something, we take refuge in being Dhamma in the present.

The impersonality of Dhamma bothers many people, because devotional religion tends to personify everything and people coming from such traditions don't feel right if they can't have some sort of personal relationship with it. I remember one time a French Catholic missionary came to stay in our monastery and practise meditation. He felt at something of a loss with Buddhism because he said it was like `cold surgery', there was no personal relationship with God. One cannot have a personal relationship with Dhamma, one cannot say `Love the Dhamma!' or `The Dhamma loves me!'; there is no need for that. We only need a personal relationship with something we are not, like our mother, father, husband or wife, something separate from us.  But we don't need to take refuge in mother or father, someone to protect us and love us and say, `I love you no matter what you do. Everything is going to be all right', and pat us on the head. The Buddha-Dhamma is a very maturing refuge, it is a religious practice that is a complete sanity or maturity, in which we are no longer seeking a mother or father, because we don't need to become anything any more. We don't need to be loved or protected by anyone any more, because we can love and protect others, and that is all that is important. We no longer have to ask or demand things from others, whether from other people or even some deity or force that we feel is separate from us and has to be prayed to and asked for guidance.  We give up all our attempts to conceive Dhamma as being this or that or anything at all, and let go of our desire to have a personal relationship with the truth. We have to be that truth, here and now. Being that truth, taking that refuge, calls for an immediate awakening, for being wise now, being Buddha, being Dhamma in the present.

The third refuge is Sa\.ngha, which means a group. `Sa\.ngha' may be the \textit{Bhikkhu-Sa\.ngha}, the order of monks, or the \textit{Ariya-Sa\.ngha}, the group of Noble Beings, all those who live virtuously, doing good and refraining from evil with bodily action and speech. Here, taking refuge in the Sa\.ngha with `\textit{Sa\.ngha\d{m} sarana\d{m} gacch\=ami}' means we take refuge in virtue, in that which is good, virtuous, kind, compassionate and generous. We don't take refuge in those things in our minds that are mean, nasty, cruel, selfish, jealous, hateful, angry -- even though admittedly that is what we often tend to do out of heedlessness, out of not reflecting, not being awake, but just reacting to conditions. Taking refuge in the Sa\.ngha means, on the conventional level, doing good and refraining from doing evil by bodily action or speech.

All of us have both good thoughts and intentions and bad ones. Sa\.nkh\=aras (conditioned phenomena) are like that: some are good and some aren't, some are neutral, some are wonderful and some are nasty. Conditions in the world are changing conditions. We can't just think the best, the most refined thoughts and feel only the best and the kindest feelings; both good and bad thoughts and feelings come and go, but we take refuge in virtue rather than in hatred. We take refuge in that in all of us that intends to do good, that is compassionate and kind and loving towards ourselves and others.

So the refuge of Sa\.ngha is a very practical refuge for day-to-day living within the human form, within this body, in relation to the bodies of other beings and the physical world that we live in. When we take this refuge we do not act in any way that causes division, disharmony, cruelty, meanness or unkindness to any living being, including ourself, our own body and mind. This is being `\textit{supa\d{t}ipanno}', one who practises well.

When we are aware and mindful, when we reflect and observe, we begin to see that acting on impulses that are cruel and selfish only brings harm and misery to ourself as well as to others. It doesn't take any great powers of observation to see that. If you've met any criminals in your life, people who have acted selfishly and evilly, you'll find them constantly frightened, obsessed, paranoid, suspicious, having to drink a lot, take drugs, keep busy, do all kinds of things, because living with themselves is so horrible. Five minutes alone with themselves without any dope or drink or anything would seem to them like eternal hell, because the kammic result of evil is so appalling mentally. Even if they're never caught by the police or sent to prison, don't think they're going to get away with anything. In fact, sometimes that is the kindest thing, to put them in prison and punish them; it makes them feel better. I was never a criminal, but I have managed to tell a few lies and do a few mean and nasty things in my lifetime, and the results were always unpleasant. Even today when I think of those things, it is not a pleasant memory, it is not something that I want to go to announce to everybody, not something that I feel joy when I think about it.

When we meditate we realize that we have to be completely responsible for how we live. In no way can we blame anyone else for anything at all. Before I started to meditate I used to blame people and society: `If only my parents had been completely wise, enlightened arahants, I would be all right. If only the United States of America had a truly wise, compassionate government that never made any mistakes, supported me completely and appreciated me fully. If only my friends were wise and encouraging and the teachers truly wise, generous and kind. If everyone around me was perfect, if society was perfect, if the world was wise and perfect, then I wouldn't have any of these problems. But all have failed me.' My parents had a few flaws and they did make a few mistakes, but now when I look back on it they didn't make very many. At the time when I was looking to blame others and I was desperately trying to think of the faults of my parents, I really had to work at it. My generation was very good at blaming everything on the United States, and that is a really easy one because the United States makes a lot of mistakes.  But when we meditate it means we can no longer get away with that kind of lying to ourselves. We suddenly realize that no matter what anyone else has done, or how unjust the society might be or what our parents might have been like, we can in no way spend the rest of our lives blaming anyone else -- that is a complete waste of time. We have to accept complete responsibility for our life, and live it. Even if we did have miserable parents, were raised in a terrible society with no opportunities, it still doesn't matter. There is no one else to blame for our suffering now but ourselves, our own ignorance, selfishness and conceit.

In the crucifixion of Jesus we can see a brilliant example of a man in pain, stripped naked, made fun of, completely humiliated and then publicly executed in the most horrible, excruciating way, yet without blaming anyone: `Forgive them, Lord, they know not what they do.' This is a sign of wisdom -- it means that even if people are crucifying us, nailing us to the cross, scourging us, humiliating us in every way, it is our aversion, self-pity, pettiness and selfishness that are the problem, the suffering. It is not even the physical pain that is the suffering, it is the aversion. Now if Jesus Christ had said, `Curse you for treating me like this!', he would have been just another criminal and would have been forgotten a few days later.

Reflect on this, because we tend easily to blame others for our suffering, and we can justify it because maybe other people are mistreating us, or exploiting us, or don't understand us or are doing dreadful things to us. We're not denying that, but we make nothing of it any more. We forgive, we let go of those memories, because taking refuge in Sa\.ngha means, here and now, doing good and refraining from doing evil by bodily action and speech.

So may you reflect on this and really see Buddha, Dhamma and Sa\.ngha as a refuge. Look on them as opportunities for reflection and consideration. It is not a matter of believing in Buddha, Dhamma, Sa\.ngha -- not a faith in concepts, but the using of symbols for mindfulness, for awakening the mind here and now; being here and now.
