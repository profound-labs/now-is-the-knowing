
We tend to overlook the ordinary. We are usually only aware of our breath when it's abnormal, like if we have asthma or when we've been running hard. But with \textit{\=an\=ap\=anasati} we take our ordinary breath as the meditation object. We don't try to make the breath long or short, or control it in any way, but simply to stay with the normal inhalation and exhalation. The breath is not something that we create or imagine; it is a natural process of our bodies that continues as long as life lasts, whether we concentrate on it or not. So it is an object that is always present; we can turn to it at any time. We don't have to have any qualifications to watch our breath. We do not even need to be particularly intelligent -- all we have to do is to be content with, and aware of, one inhalation and exhalation. Wisdom does not come from studying great theories and philosophies, but from observing the ordinary.

The breath lacks any exciting quality or anything fascinating about it, and so we can become very restless and averse to it. Our desire is always to `get' something, to find something that will interest and absorb us without any effort on our part. If we hear some music, we don't think, `I must concentrate on this fascinating and exciting rhythmic music'; we can't stop ourselves, because the rhythm is so compelling that it pulls us in. The rhythm of our normal breathing is not interesting or compelling; it is tranquillizing, and most beings aren't used to tranquillity. Most people like the idea of peace, but find the actual experience of it disappointing or frustrating. They desire stimulation, something that will draw them into itself. With \textit{\=an\=ap\=anasati} we stay with an object that is quite neutral -- we don't have any strong feelings of liking or disliking for our breath -- and just note the beginning of an inhalation, its middle and its end, then the beginning of an exhalation, its middle and end. The gentle rhythm of the breath, being slower than the rhythm of thought, takes us to tranquillity; we begin to stop thinking. But we don't try to get anything from the meditation, to get \textit{sam\=adhi} or get \textit{jh\=ana}, because when the mind is trying to achieve or attain things, rather than just being humbly content with one breath, then it doesn't slow down and become calm, and we become frustrated.

At first the mind wanders off. Once we are aware that we have wandered off the breath, then we very gently return to it. We use the attitude of being very, very patient and always willing to begin again. Our minds are not used to being held down; they have been taught to associate one thing with another and form opinions about everything. Being accustomed to using our intelligence and ability to think in clever ways, we tend to become very tense and restless when we can't do that, and when we practise \textit{\=an\=ap\=anasati} we feel resistance, a resentment of it. We are like a wild horse when it is first harnessed, getting angry with the things that bind it.

When the mind wanders we get upset and discouraged, negative and averse to the whole thing. If out of frustration, we try by sheer will to force the mind to be tranquil, we can only keep it up for a short while and then the mind is off somewhere else. So the right attitude to \textit{\=an\=ap\=anasati} is being very patient, having all the time in the world, letting go or discarding all worldly, personal or financial problems. During this time there is nothing we have to do except watch our breath.

If the mind wanders on the in-breath, then put more effort into the inhalation. If the mind wanders on the exhalation, then put more effort into that. Keep bringing it back. Always be willing to start anew. At the start of each new day, at the beginning of each inhalation, cultivate the beginner's mind, carrying nothing from the old to the new, leaving no traces, like a big bonfire.

One inhalation and the mind wanders, so we bring it back again -- and that itself is a moment of mindfulness. We are training the mind like a good mother trains her child. A little child doesn't know what it is doing; it just wanders off; and if the mother gets angry with it and spanks and beats it, the child becomes terrified and neurotic. A good mother will just leave the child, keeping an eye on it, and if it wanders she will bring it back. Having that kind of patience, we're not trying to bash away at ourselves, hating ourselves, hating our breath, hating everybody, getting upset because we can't get tranquil with \textit{\=an\=ap\=anasati}.

Sometimes we get too serious about everything, totally lacking in joy and happiness, with no sense of humour, just repressing everything. Gladden the mind, put a smile on your dial! Be relaxed and at ease, without the pressure of having to achieve anything special -- nothing to attain, no big deal, nothing special. And what can you say you have done today to earn your board and keep? Just one mindful inhalation? Crazy! But that is more than most people can say of their day.

We're not battling the forces of evil. If you feel averse to \textit{\=an\=ap\=anasati}, then note that too. Don't feel it's something you have to do, but let it be a pleasure, something you really enjoy doing. When you think `I can't do it', recognize that as resistance, fear or frustration and then relax. Don't make this practice into a difficult thing, a burdensome task.  When I was first ordained I was dead serious, very grim and solemn about myself, like a dried-up old stick, and I used to get in terrible states, thinking, `I've got to... I've got to....' At those times I learned to contemplate peace. Doubts and restlessness, discontent and aversion -- soon I was able to reflect on peace, saying the word over and over, hypnotizing myself to relax. The self-doubts would start coming -- `I'm getting nowhere with this, it's useless, I want to get something' -- and I was able to be peaceful with that. This is one method that you can use. So when we're tense, we relax and then we resume \textit{\=an\=ap\=anasati}.

\looseness=1
At first we feel hopelessly clumsy, like when we're learning to play the guitar - when we first start playing, our fingers are so clumsy it seems hopeless, but when we've done it for some time we gain skill and it is quite easy. We're learning to witness what's going on in our mind, so we can know when we're getting restless and tense or when we're getting dull. We recognize that: we're not trying to convince ourselves that it's otherwise, we're fully aware of the way things are. We sustain effort for one inhalation. If we can't do that, then we sustain it for half an inhalation at least. In this way we're not trying to become perfect all at once. We don't have to do everything just right according to some idea of how it should be, but we work with the problems that are there. If we have a scattered mind, then it's wisdom to recognize the mind that goes all over the place -- that is insight. To think that we shouldn't be that way, to hate ourselves or feel discouraged because that is the way we happen to be -- that is ignorance.

We don't start from where a perfect yogi is, we're not doing Iyengar postures before we can bend over and touch our toes. That is the way to ruin ourselves. We may look at all the postures in the \textit{Light on Yoga} book and see Iyengar wrapping his legs around his neck in all kinds of amazing postures, but if we try to do them ourselves they'll cart us off to hospital. So we start from just trying to bend a little more from the waist, examining the pain and resistance to it, learning to stretch gradually. The same with \textit{\=an\=ap\=anasati}: we recognize the way it is now and start from there, we sustain our attention a little longer, and we begin to understand what concentration is. Don't make Superman resolutions when you're not Superman. You say, `I'm going to sit and watch my breath all night,' and then when you fail you become angry. Set periods that you know you can do. Experiment, work with the mind until you know how to put forth effort and how to relax.

We have to learn to walk by falling down. Look at babies: I've never seen one that could walk straightaway. Babies learn to walk by crawling, by holding onto things, by falling down and then pulling themselves up again. It is the same with meditation. We learn wisdom by observing ignorance, by making a mistake, reflecting and keeping going. If we think about it too much, it seems hopeless. If babies thought a lot, they'd never learn to walk, because when you watch a child trying to walk it seems hopeless, doesn't it? When we think about it, meditation can seem completely hopeless, but we just keep doing it. It is easy when we're full of enthusiasm, really inspired with the teacher and the teaching -- but enthusiasm and inspiration are impermanent conditions, they take us to disillusionment and boredom.

When we're bored, we really have to put effort into the practice. When we're bored, we want to turn away and be reborn into something fascinating and exciting. But for insight and wisdom, we have to endure patiently through the troughs of disillusionment and depression. It is only in this way that we can stop reinforcing the cycles of habit, and come to understand cessation, come to know the silence and emptiness of the mind.

If we read books about not putting any effort into things, just letting everything happen in a natural, spontaneous way, then we tend to start thinking that all we have to do is lounge about -- and then we lapse into a dull passive state. In my own practice, when I lapsed into dull states I came to see the importance of putting effort into physical posture. I saw that there was no point in making effort in a merely passive way. I would pull the body up straight, push out the chest and put energy into the sitting posture; or else I would do head stands or shoulder stands. Even though in the early days I didn't have a tremendous amount of energy, I still managed to do something requiring effort. I would learn to sustain it for a few seconds and then I would lose it again, but that was better than doing nothing at all.

The more we take the easy way, the path of least resistance, the more we just follow our desires, the more the mind becomes sloppy, heedless and confused. It is easy to think, easier to sit and think all the time than not to think -- it is a habit we've acquired. Even the thought, `I shouldn't think,' is just another thought. To avoid thought we have to be mindful of it, to put forth effort by watching and listening, by being attentive to the flow in our minds. Rather than thinking about our mind, we watch it. Rather than just getting caught in thoughts, we keep recognizing them. Thought is movement, it is energy, it comes and goes, it is not a \mbox{permanent} \mbox{condition} of the mind. Without evaluating or analyzing, when we simply recognize thought as thought it begins to slow down and stop. This isn't annihilation; this is allowing things to cease. It is compassion. As the habitual obsessive thinking begins to fade, great spaces we never knew were there begin to appear.

We are slowing everything down by absorbing into the natural breath, calming the kammic formations, and this is what we mean by \textit{samatha} or tranquillity: coming to a point of calm. The mind becomes malleable, supple and flexible, and the breathing can become very fine. But we only carry the \textit{samatha} practice to the point of \textit{upac\=ara sam\=adhi} (approaching concentration), we don't try to absorb completely into the object and enter \textit{jh\=ana}. At this point we are still aware of both the object and its periphery. The extreme kinds of mental agitation have diminished considerably, but we can still operate using wisdom.

With our wisdom faculty still functioning, we investigate, and this is \textit{vipassan\=a} -- looking into and seeing the nature of whatever we experience, its impermanence, unsatisfactoriness and impersonality. \textit{Anicca}, \textit{dukkha} and \textit{anatt\=a} are not concepts we believe in, but things we can observe. We investigate the beginning of an inhalation and its ending. We observe what a beginning is, not thinking about what it is but observing, aware with bare attention at the beginning of an inhalation and its end. The body breathes all on its own: the in-breath conditions the out-breath and the out-breath conditions the in-breath, we can't control anything. \mbox{Breathing belongs} to nature, it doesn't belong to us -- it is not-self. When we see this we are doing \textit{vipassan\=a}.

The sort of knowledge we gain from Buddhist meditation is humbling. Ajahn Chah calls it the earth-worm knowledge -- it doesn't make you arrogant, it doesn't puff you up, it doesn't make you feel that you are anything, or that you have attained anything. In worldly terms, this practice doesn't seem very important or necessary. Nobody is ever going to write a newspaper headline: `At eight o'clock this evening Venerable Sumedho had an inhalation'! To some people thinking about how to solve all the world's problems might seem very important - how to help all the people in the Third World, how to set the world right. Compared with these things, watching our breath seems insignificant, and most people think, `Why waste time doing that?' People have confronted me about this, saying: `What are you monks doing sitting there? What are you doing to help humanity? You're just selfish, you expect people to give you food while you just sit there and watch your breath. You're running away from the real world.'

But what is the real world? Who is really running away, and from what? What is there to face? We find that what people call the real world is the world they believe in, the world that they are committed to, or the world that they know and are familiar with. But that world is a condition of mind. Meditation is actually confronting the real world, recognizing and acknowledging it as it really is, rather than believing in it or justifying it or trying mentally to annihilate it. Now, the real world operates on the same pattern of arising and passing as the breath. We're not theorizing about the nature of things, taking philosophical ideas from others and trying to rationalize with them, but by watching our breath we're actually observing the way nature is. When we're watching our breath we're actually watching nature; through understanding the nature of the breath, we can understand the nature of all conditioned phenomena. If we tried to understand all conditioned phenomena in their infinite variety, quality, different time span and so on, it would be too complex; our minds wouldn't be able to handle it. We have to learn from simplicity.  

So with a tranquil mind we become aware of the cyclical pattern, we see that all that arises passes away. That cycle is what is called \textit{sa\d{m}s\=ara}, the wheel of birth and death. We observe the `samsaric` cycle of the breath. We inhale and then we exhale: we can't have only inhalations or only exhalations, the one conditions the other. It would be absurd to think, `I only want to inhale. I don't want to exhale. I'm giving up exhalation. My life will be just one constant inhalation'. That would be absolutely ridiculous. If I said that to you, you'd think I was crazy; but that is what most people do. How foolish people are when they want only to attach to excitement, pleasure, youth, beauty and vigour. `I only want beautiful things and I'm not going to have anything to do with the ugly. I want pleasure and delight and creativity, but I don't want any boredom or depression.' It is the same kind of madness as if you were to hear me saying, `I can't stand inhalations. I'm not going to have them any more.' When we observe that attachment to beauty, sensual pleasures and love will always lead to despair, then our attitude becomes one of detachment. That doesn't mean annihilation or any desire to destroy, but simply letting go, non-attachment. We don't seek perfection in any part of the cycle, but see that perfection lies in the cycle as a whole: it includes old age, sickness and death. What arises in the uncreated reaches its peak and then returns to the uncreated, and that is perfection.

As we start to see that all sa\.nkh\=aras have this pattern of arising and passing away, we begin to go inwards to the unconditioned, the peace of the mind, its silence. We begin to experience \textit{su\~n\~nat\=a} or emptiness, which is not oblivion or nothingness, but a clear and vibrant stillness. We can actually turn to the emptiness rather than to the conditions of the breath and mind. Then we have a perspective on the conditions and don't just blindly react to them any more.

There is the conditioned, the unconditioned and the knowing. What is the knowing? Is it memory? Is it consciousness? Is it `me'? I've never been able to find out, but I can be aware. In Buddhist meditation we stay with the knowing: being aware, being awake, being Buddha in the present, knowing that whatever arises passes away and is not-self. We apply this knowing to everything, both the conditioned and the unconditioned. It is transcending -- being awake rather than trying to escape -- and it is all in our ordinary activity. We have the four normal postures of sitting, standing, walking and lying down -- we don't have to stand on our heads or do back-flips or anything. We use four normal postures and the ordinary breathing, because we are moving towards that which is most ordinary, the unconditioned. Conditions are extraordinary, but the peace of the mind, the unconditioned, is so ordinary that nobody ever notices it. It is there all the time, but we don't ever notice it because we're attached to the mysterious and the fascinating. We get caught up in the things that arise and pass away, the things that stimulate and depress. We get caught up in the way things seem to be -- and forget. But now we're going back to that source in meditation, to that peace, in that position of knowing. Then the world is understood for what it is, and we are no longer deluded by it.

The realization of \textit{sa\d{m}s\=ara} is the condition of Nibb\=ana. As we recognize the cycles of habit and are no longer deluded by them or their qualities, we realize Nibb\=ana. The Buddha-knowing is of just two things: the conditioned and the unconditioned. It is an immediate recognition of how things are right now, without grasping or attachment. At this moment we can be aware of the conditions of the mind, feelings in the body, what we're seeing, hearing, tasting, touching, smelling and thinking, and also of the emptiness of the mind. The conditioned and the unconditioned are what we can realize.

So the Buddha's teaching is a very direct teaching. Our practice is not `to become enlightened', but to be in the knowing, now.

